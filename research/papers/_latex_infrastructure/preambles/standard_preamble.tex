% Standard Preamble for Neural Assembly Papers
% Include this in all papers for consistent styling

% ============================================================================
% MATH PACKAGES
% ============================================================================
\usepackage{amsmath}        % Advanced math
\usepackage{amssymb}        % Math symbols
\usepackage{amsthm}         % Theorem environments
\usepackage{mathtools}      % Extensions to amsmath

% ============================================================================
% GRAPHICS AND FIGURES
% ============================================================================
\usepackage{graphicx}       % Include graphics
\usepackage{xcolor}         % Colors
\usepackage{tikz}           % Vector graphics
\usepackage{pgfplots}       % Plots
\pgfplotsset{compat=1.18}

% ============================================================================
% TABLES
% ============================================================================
\usepackage{booktabs}       % Professional tables
\usepackage{multirow}       % Multi-row cells
\usepackage{array}          % Extended array/tabular

% ============================================================================
% ALGORITHMS
% ============================================================================
\usepackage{algorithm}      % Algorithm environment
\usepackage{algorithmic}    % Algorithm pseudocode
% Or use algorithmicx for more control:
% \usepackage{algpseudocode}

% ============================================================================
% CODE LISTINGS
% ============================================================================
\usepackage{listings}       % Code listings
\lstset{
    basicstyle=\ttfamily\small,
    breaklines=true,
    frame=single,
    numbers=left,
    numberstyle=\tiny,
    commentstyle=\color{gray},
    keywordstyle=\color{blue}
}

% ============================================================================
% BIBLIOGRAPHY
% ============================================================================
\usepackage{natbib}         % Natural bibliography
\bibliographystyle{plainnat}

% ============================================================================
% HYPERLINKS AND REFERENCES
% ============================================================================
\usepackage{hyperref}       % Hyperlinks
\hypersetup{
    colorlinks=true,
    linkcolor=blue,
    citecolor=blue,
    urlcolor=blue,
}
\usepackage{cleveref}       % Smart references

% ============================================================================
% UTILITY PACKAGES
% ============================================================================
\usepackage{xspace}         % Smart spacing
\usepackage{enumitem}       % Better lists
\usepackage{subcaption}     % Subfigures

% ============================================================================
% THEOREM ENVIRONMENTS
% ============================================================================
\theoremstyle{plain}
\newtheorem{theorem}{Theorem}[section]
\newtheorem{lemma}[theorem]{Lemma}
\newtheorem{corollary}[theorem]{Corollary}
\newtheorem{proposition}[theorem]{Proposition}

\theoremstyle{definition}
\newtheorem{definition}[theorem]{Definition}
\newtheorem{example}[theorem]{Example}

\theoremstyle{remark}
\newtheorem{remark}[theorem]{Remark}
\newtheorem{note}[theorem]{Note}

% ============================================================================
% GENERAL NOTATION
% ============================================================================

% Sets
\newcommand{\R}{\mathbb{R}}              % Real numbers
\newcommand{\N}{\mathbb{N}}              % Natural numbers
\newcommand{\Z}{\mathbb{Z}}              % Integers
\newcommand{\C}{\mathbb{C}}              % Complex numbers

% Probability
\newcommand{\Prob}{\mathbb{P}}           % Probability
\newcommand{\Expect}{\mathbb{E}}         % Expectation
\newcommand{\Var}{\mathrm{Var}}          % Variance
\newcommand{\Cov}{\mathrm{Cov}}          % Covariance

% Complexity
\newcommand{\bigO}{\mathcal{O}}          % Big-O notation
\newcommand{\bigOmega}{\Omega}           % Big-Omega notation
\newcommand{\bigTheta}{\Theta}           % Big-Theta notation

% Vectors and matrices
\newcommand{\vect}[1]{\mathbf{#1}}       % Vectors (bold)
\newcommand{\matr}[1]{\mathbf{#1}}       % Matrices (bold capital)

% Operators
\DeclareMathOperator*{\argmax}{arg\,max}
\DeclareMathOperator*{\argmin}{arg\,min}
\DeclareMathOperator{\rank}{rank}
\DeclareMathOperator{\trace}{tr}
\DeclareMathOperator{\diag}{diag}

% ============================================================================
% CUSTOM COMMANDS FOR READABILITY
% ============================================================================

% Use \eg, \ie, \etal with proper spacing
\newcommand{\eg}{e.g.,\xspace}
\newcommand{\ie}{i.e.,\xspace}
\newcommand{\etal}{et al.\xspace}
\newcommand{\vs}{vs.\xspace}

% ============================================================================
% END OF STANDARD PREAMBLE
% ============================================================================
